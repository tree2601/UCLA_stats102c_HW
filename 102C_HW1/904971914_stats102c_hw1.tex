\PassOptionsToPackage{unicode=true}{hyperref} % options for packages loaded elsewhere
\PassOptionsToPackage{hyphens}{url}
%
\documentclass[
]{article}
\usepackage{lmodern}
\usepackage{amssymb,amsmath}
\usepackage{ifxetex,ifluatex}
\ifnum 0\ifxetex 1\fi\ifluatex 1\fi=0 % if pdftex
  \usepackage[T1]{fontenc}
  \usepackage[utf8]{inputenc}
  \usepackage{textcomp} % provides euro and other symbols
\else % if luatex or xelatex
  \usepackage{unicode-math}
  \defaultfontfeatures{Scale=MatchLowercase}
  \defaultfontfeatures[\rmfamily]{Ligatures=TeX,Scale=1}
\fi
% use upquote if available, for straight quotes in verbatim environments
\IfFileExists{upquote.sty}{\usepackage{upquote}}{}
\IfFileExists{microtype.sty}{% use microtype if available
  \usepackage[]{microtype}
  \UseMicrotypeSet[protrusion]{basicmath} % disable protrusion for tt fonts
}{}
\makeatletter
\@ifundefined{KOMAClassName}{% if non-KOMA class
  \IfFileExists{parskip.sty}{%
    \usepackage{parskip}
  }{% else
    \setlength{\parindent}{0pt}
    \setlength{\parskip}{6pt plus 2pt minus 1pt}}
}{% if KOMA class
  \KOMAoptions{parskip=half}}
\makeatother
\usepackage{xcolor}
\IfFileExists{xurl.sty}{\usepackage{xurl}}{} % add URL line breaks if available
\IfFileExists{bookmark.sty}{\usepackage{bookmark}}{\usepackage{hyperref}}
\hypersetup{
  pdftitle={904971914\_stats102c\_hw1},
  pdfauthor={Xiaoshu Luo 904971914},
  pdfborder={0 0 0},
  breaklinks=true}
\urlstyle{same}  % don't use monospace font for urls
\usepackage[margin=1in]{geometry}
\usepackage{color}
\usepackage{fancyvrb}
\newcommand{\VerbBar}{|}
\newcommand{\VERB}{\Verb[commandchars=\\\{\}]}
\DefineVerbatimEnvironment{Highlighting}{Verbatim}{commandchars=\\\{\}}
% Add ',fontsize=\small' for more characters per line
\usepackage{framed}
\definecolor{shadecolor}{RGB}{248,248,248}
\newenvironment{Shaded}{\begin{snugshade}}{\end{snugshade}}
\newcommand{\AlertTok}[1]{\textcolor[rgb]{0.94,0.16,0.16}{#1}}
\newcommand{\AnnotationTok}[1]{\textcolor[rgb]{0.56,0.35,0.01}{\textbf{\textit{#1}}}}
\newcommand{\AttributeTok}[1]{\textcolor[rgb]{0.77,0.63,0.00}{#1}}
\newcommand{\BaseNTok}[1]{\textcolor[rgb]{0.00,0.00,0.81}{#1}}
\newcommand{\BuiltInTok}[1]{#1}
\newcommand{\CharTok}[1]{\textcolor[rgb]{0.31,0.60,0.02}{#1}}
\newcommand{\CommentTok}[1]{\textcolor[rgb]{0.56,0.35,0.01}{\textit{#1}}}
\newcommand{\CommentVarTok}[1]{\textcolor[rgb]{0.56,0.35,0.01}{\textbf{\textit{#1}}}}
\newcommand{\ConstantTok}[1]{\textcolor[rgb]{0.00,0.00,0.00}{#1}}
\newcommand{\ControlFlowTok}[1]{\textcolor[rgb]{0.13,0.29,0.53}{\textbf{#1}}}
\newcommand{\DataTypeTok}[1]{\textcolor[rgb]{0.13,0.29,0.53}{#1}}
\newcommand{\DecValTok}[1]{\textcolor[rgb]{0.00,0.00,0.81}{#1}}
\newcommand{\DocumentationTok}[1]{\textcolor[rgb]{0.56,0.35,0.01}{\textbf{\textit{#1}}}}
\newcommand{\ErrorTok}[1]{\textcolor[rgb]{0.64,0.00,0.00}{\textbf{#1}}}
\newcommand{\ExtensionTok}[1]{#1}
\newcommand{\FloatTok}[1]{\textcolor[rgb]{0.00,0.00,0.81}{#1}}
\newcommand{\FunctionTok}[1]{\textcolor[rgb]{0.00,0.00,0.00}{#1}}
\newcommand{\ImportTok}[1]{#1}
\newcommand{\InformationTok}[1]{\textcolor[rgb]{0.56,0.35,0.01}{\textbf{\textit{#1}}}}
\newcommand{\KeywordTok}[1]{\textcolor[rgb]{0.13,0.29,0.53}{\textbf{#1}}}
\newcommand{\NormalTok}[1]{#1}
\newcommand{\OperatorTok}[1]{\textcolor[rgb]{0.81,0.36,0.00}{\textbf{#1}}}
\newcommand{\OtherTok}[1]{\textcolor[rgb]{0.56,0.35,0.01}{#1}}
\newcommand{\PreprocessorTok}[1]{\textcolor[rgb]{0.56,0.35,0.01}{\textit{#1}}}
\newcommand{\RegionMarkerTok}[1]{#1}
\newcommand{\SpecialCharTok}[1]{\textcolor[rgb]{0.00,0.00,0.00}{#1}}
\newcommand{\SpecialStringTok}[1]{\textcolor[rgb]{0.31,0.60,0.02}{#1}}
\newcommand{\StringTok}[1]{\textcolor[rgb]{0.31,0.60,0.02}{#1}}
\newcommand{\VariableTok}[1]{\textcolor[rgb]{0.00,0.00,0.00}{#1}}
\newcommand{\VerbatimStringTok}[1]{\textcolor[rgb]{0.31,0.60,0.02}{#1}}
\newcommand{\WarningTok}[1]{\textcolor[rgb]{0.56,0.35,0.01}{\textbf{\textit{#1}}}}
\usepackage{graphicx,grffile}
\makeatletter
\def\maxwidth{\ifdim\Gin@nat@width>\linewidth\linewidth\else\Gin@nat@width\fi}
\def\maxheight{\ifdim\Gin@nat@height>\textheight\textheight\else\Gin@nat@height\fi}
\makeatother
% Scale images if necessary, so that they will not overflow the page
% margins by default, and it is still possible to overwrite the defaults
% using explicit options in \includegraphics[width, height, ...]{}
\setkeys{Gin}{width=\maxwidth,height=\maxheight,keepaspectratio}
\setlength{\emergencystretch}{3em}  % prevent overfull lines
\providecommand{\tightlist}{%
  \setlength{\itemsep}{0pt}\setlength{\parskip}{0pt}}
\setcounter{secnumdepth}{-2}
% Redefines (sub)paragraphs to behave more like sections
\ifx\paragraph\undefined\else
  \let\oldparagraph\paragraph
  \renewcommand{\paragraph}[1]{\oldparagraph{#1}\mbox{}}
\fi
\ifx\subparagraph\undefined\else
  \let\oldsubparagraph\subparagraph
  \renewcommand{\subparagraph}[1]{\oldsubparagraph{#1}\mbox{}}
\fi

% set default figure placement to htbp
\makeatletter
\def\fps@figure{htbp}
\makeatother


\title{904971914\_stats102c\_hw1}
\author{Xiaoshu Luo 904971914}
\date{}

\begin{document}
\maketitle

\hypertarget{question-1}{%
\subsection{Question 1}\label{question-1}}

Since X and Y are independent, X\^{}2 and Y\^{}2 are also independent:

\[
\begin{eqnarray}
Var(XY) 
& = & E((XY)^2)-E^2(XY)\\
& = & E(X^2)\times E(Y^2)-E(XY)\times E(XY)\\
& = &  (E^2(X)+Var(X)) \times (E^2(Y)+Var(Y)) - E^2(X)E^2(Y)\\
& = &  E^2(X)\times Var(Y) + E^2(Y)\times Var(X) + Var(X) \times Var(Y) 
\end{eqnarray}
\]

\hypertarget{question-2}{%
\subsection{Question 2}\label{question-2}}

{[} \begin{eqnarray}

X & \sim & Geo(0.6)\\
P(X \leq k) & \approx & 0.9\\
F_X(k)& \approx & 0.9\\
1-(1-0.6)^{k+1}& \approx & 0.9\\
k &\approx &  1.5\\
k &=& 1\Rightarrow P(X \leq k)=0.84\\
k&=&2\Rightarrow P(X \leq k)=0.93\\
k &=& 2


\end{eqnarray} {]}

\hypertarget{question-3}{%
\subsection{Question 3}\label{question-3}}

\textbf{(a)} Let T be the time between successive calls. {[}
\begin{eqnarray}

T &\sim& exp(\lambda),\lambda=6\\
E(T) &=& 1/\lambda = 1/6
 
\end{eqnarray} {]}

The average times is 1/6 minute.

\textbf{(b)} {[} \begin{eqnarray}
N_1 &\sim& Poi(\lambda),\lambda=6\\
P(N_1 = 5) &=& \frac{6^5e^{-6}}{5!} \\
&=& 0.1606


\end{eqnarray} {]}

\textbf{(c)} {[} \begin{eqnarray}
P(N_1 < 5) &=& \sum_{k=0}^{4}\frac{6^ke^{-6}}{k!} \\
P(N_1 < 5) &=& 0.2851

\end{eqnarray} {]}

\textbf{(d))} Let T be the waiting time in seconds. {[} \begin{eqnarray}
T&\sim& poi(\lambda'),\lambda'=6/60=0.1\\
P(T \leq 5) &=& F_T(5)=1-{e^{-0.1 * 5}} \\
&=&0.3935

\end{eqnarray} {]}

\textbf{(e)}

\begin{Shaded}
\begin{Highlighting}[]
\KeywordTok{set.seed}\NormalTok{(}\DecValTok{904971914}\NormalTok{)}
\NormalTok{x<-}\KeywordTok{rexp}\NormalTok{(}\DataTypeTok{n=}\DecValTok{1000}\NormalTok{,}\DataTypeTok{rate =}\FloatTok{0.1}\NormalTok{)}
\NormalTok{q3e<-}\KeywordTok{ecdf}\NormalTok{(x)}

\NormalTok{z <-}\KeywordTok{seq}\NormalTok{(}\DecValTok{0}\NormalTok{,}\DecValTok{80}\NormalTok{,}\FloatTok{0.1}\NormalTok{)}
\NormalTok{zy <-}\KeywordTok{q3e}\NormalTok{(z)}
\NormalTok{zci<-}\KeywordTok{sqrt}\NormalTok{((}\KeywordTok{q3e}\NormalTok{(zy)}\OperatorTok{*}\NormalTok{(}\DecValTok{1}\OperatorTok{-}\KeywordTok{q3e}\NormalTok{(zy)))}\OperatorTok{/}\DecValTok{1000}\NormalTok{)}\OperatorTok{*}\FloatTok{1.96} 

\KeywordTok{plot}\NormalTok{(}\DataTypeTok{x=}\NormalTok{z,}\DataTypeTok{y=}\NormalTok{zy, }\DataTypeTok{ylab =} \KeywordTok{expression}\NormalTok{(}\KeywordTok{P}\NormalTok{(X }\OperatorTok{<=}\StringTok{ }\NormalTok{x)), }\DataTypeTok{main =} \StringTok{"EDF with confidence interval"}\NormalTok{,}\DataTypeTok{type=}\StringTok{'l'}\NormalTok{,}\DataTypeTok{las =} \DecValTok{1}\NormalTok{)}

\NormalTok{U<-}\StringTok{ }\NormalTok{zy}\OperatorTok{+}\NormalTok{zci}
\NormalTok{L<-zy}\OperatorTok{-}\NormalTok{zci}
\KeywordTok{lines}\NormalTok{(z,U,}\DataTypeTok{col=}\StringTok{'red'}\NormalTok{)}
\KeywordTok{lines}\NormalTok{(z,L,}\DataTypeTok{col=}\StringTok{'red'}\NormalTok{)}
\end{Highlighting}
\end{Shaded}

\includegraphics{904971914_stats102c_hw1_files/figure-latex/q3e-1.pdf}
The red lines are the confidence intervals.

\hypertarget{question-4}{%
\subsection{Question 4}\label{question-4}}

\textbf{(a)} \[
\begin{eqnarray}
Y&=&F^{-1}(u)\\
P(Y \leq x)&=& P(F^{-1}(u) \leq  x)\\
&=& P(u \leq F(x))\\
&=& F_U(F(x)) \\
&=& \int _0 ^ {F(x)} 1 du \\
&=& F_{X}(x) 
\end{eqnarray}
\]

\textbf{(b)} {[} \begin{eqnarray}
Let \space \space Y&=&F(x),0 \leq Y\leq 1  \\
P(Y \leq y)&=&P(F(x) \leq y)\\
&=&P(X \leq F^{-1}(y))\\
&=& F(F^{-1}(y))\\
&=& y, \space y \in[0,1]

\end{eqnarray} {]} Therefore, Y has the same CDF as U \textasciitilde{}
{[}0,1{]},so F(x) \textasciitilde{} {[}0,1{]}.

\hypertarget{question-5}{%
\subsection{Question 5}\label{question-5}}

\textbf{(a)} {[} \begin{eqnarray}
F(x) &=&\int_{-\infty}^xf(t)dt\\
&=& \int_{\eta}^x\lambda e^{-\lambda(t-\eta)}dt\\
&=& - e^{\lambda\eta} (e^{-\lambda x}-e^{-\lambda \eta})\\
&=& 1-e^{\lambda(\eta-x)}

\end{eqnarray} {]}

\textbf{(b)} {[} \begin{eqnarray}
F(x) &=& u\\
1-e^{\lambda(\eta - x)} &=& u \\
\lambda\eta - \lambda x &=& log(1-u)\\
x = F^{-1}(u) &=& \eta - \frac {log(1-u)} \lambda\\


\end{eqnarray} {]}

\textbf{(c)}

\begin{Shaded}
\begin{Highlighting}[]
\NormalTok{q5c <-}\ControlFlowTok{function}\NormalTok{(lambda,eta,n)\{}
\CommentTok{#This function generates random numbers from the two-parameter exponential distribution.}
\CommentTok{#lambda and eta are parameters and n specifies how many numbers to generate.}
\CommentTok{#The function gives a number or a vector of numbers generated from the distribution. }
\NormalTok{  x <-}\StringTok{ }\KeywordTok{runif}\NormalTok{(n)}
\NormalTok{  y <-}\StringTok{ }\NormalTok{eta}\OperatorTok{-}\NormalTok{(}\KeywordTok{log}\NormalTok{(}\DecValTok{1}\OperatorTok{-}\NormalTok{x)}\OperatorTok{/}\NormalTok{lambda)}
\NormalTok{  y}
\NormalTok{\}}
\end{Highlighting}
\end{Shaded}

\textbf{(d)}

\begin{Shaded}
\begin{Highlighting}[]
\NormalTok{q5d <-}\KeywordTok{q5c}\NormalTok{(}\DataTypeTok{lambda=}\DecValTok{4}\NormalTok{,}\DataTypeTok{eta=}\DecValTok{1}\NormalTok{,}\DataTypeTok{n=}\DecValTok{10000}\NormalTok{)}
\KeywordTok{hist}\NormalTok{(q5d,}\DataTypeTok{prob=}\NormalTok{T)}
\NormalTok{y<-}\KeywordTok{seq}\NormalTok{(}\DecValTok{0}\NormalTok{,}\DecValTok{10}\NormalTok{,}\FloatTok{0.01}\NormalTok{)}
\KeywordTok{lines}\NormalTok{(y,}\DecValTok{4}\OperatorTok{*}\KeywordTok{exp}\NormalTok{(}\OperatorTok{-}\DecValTok{4}\OperatorTok{*}\NormalTok{(y}\DecValTok{-1}\NormalTok{)))}
\end{Highlighting}
\end{Shaded}

\includegraphics{904971914_stats102c_hw1_files/figure-latex/q5d-1.pdf}

\textbf{(e)} The sample quartiles is very similar to the theoretical
exponential quartiles.

\end{document}
